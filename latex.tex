\documentclass[12pt,a10paper]{report}
\usepackage{graphicx}
\usepackage[hmargin=5cm,
vmargin=3cm]{geometry}
\title{\textbf{Shri G. S. Institute of Technology and Science}}
\author{\Huge{CO24997 : programming practices}}
{\huge{\date{\today}}}

\begin{document}
\maketitle
\title{\textbf{\large{MINI PROJECT :  UNIT CONVERTER :}}}
\large{starting date : 13/11/22 }\newline
\large{ending date : 21/11/22}\newline\newline
\textbf{\large{ The project is coded in two languages c and java }}\newline\newline
\section{\textbf{UNIT CONVERTER :}}
\large{project is made on unit converter.Unit converter enables conversion and switching the base unit to converting unit. In this project we use 5 quantities for converting into unit, for ex.  meter into kilometer and million to crores.}\newline\newline
\huge{ five quantities are used in this project :}\newline\newline
\large{1. mass : kilogram and gram} \newline\newline
\large{2.temperature  : kelvin and  celcius}  \newline\newline
\large{3.time : seconds and  hours}\newline\newline
\large{4.currency :   million and  crores }\newline\newline
\large{5. length : kilometer and  meter }\newline\newline
\section{\textbf{IN C language :}}
\large{no. of code line is : {\textbf{205}}
In unit converter we use 10 user define function are:\newline\newline
1. float getgram(float kilogram);\newline\newline
2. float getkilogram(float gram);\newline\newline
3. float getmillion(float crores);\newline\newline
4. float getcrores(float million);\newline\newline
5. float getkelvin(float celcius);\newline\newline
6. float getcelcius(float kelvin);\newline\newline
7. float gethours(float seconds);\newline\newline
8. float getseconds(float hours);\newline\newline
9. float getkilometer(float meter);\newline\newline
10. float getmeter(float kilometer); \newline\newline
\pagebreak
\graphicspath{{./deepikamandroniyahp/} }
\subsection{debugging :}
\textbf{The screenshoot of debbuging of unit converter program are :}\newline\newline
\includegraphics[scale=0.9]{debugging1}
\pagebreak
\subsection{profilling :}
\textbf{The screenshoots of profiling of unit converter program are :}\newline\newline
\includegraphics[scale=0.7]{profiling 6}

\includegraphics[scale=0.7]{profiling 7}
\includegraphics[scale=0.7]{profiling 8}
\includegraphics[scale=0.7]{profiling 9}
\includegraphics[scale=0.7]{profiling 10}
\pagebreak
\subsection{CODE OUTPUT :}
\textbf{The screenshoot of  c code output are :}\newline\newline
\includegraphics[scale=0.3]{c output}
\pagebreak
\section{\textbf{In java language :}}
\large{no. of code line is : {\textbf{168}}\newline\newline
The project made with 10 user define function in c language and now we convert into java language:\newline\newline

\subsection{CODE OUTPUT :}
\textbf{The screenshoot of  java code output are :}\newline\newline
\includegraphics[scale=0.3]{java output}\newline\newline\newline\newline
\centering{\textbf{END OF REPORT}}




\end{document}